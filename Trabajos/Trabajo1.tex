\documentclass{article}
\usepackage{url}
\usepackage{amsmath}
\usepackage{bm}
\pagestyle{empty}
\setlength{\textwidth}{6in}
\setlength{\oddsidemargin}{0in}
\setlength{\topmargin}{-.75in}
\setlength{\textheight}{9.5in}

\usepackage{color}
\definecolor{myred}{rgb}{0.6,0,0}

\usepackage[utf8]{inputenc}
\usepackage[spanish]{babel}
%% Place here your \newcommand's and \renewcommand's. Some examples already included.
%%
\renewcommand{\le}{\leqslant}
\renewcommand{\ge}{\geqslant}
\renewcommand{\emptyset}{\ensuremath{\varnothing}}
\newcommand{\ds}{\displaystyle}
\newcommand{\R}{\ensuremath{\mathbb{R}}}
\newcommand{\Q}{\ensuremath{\mathbb{Q}}}
\newcommand{\Z}{\ensuremath{\mathbb{Z}}}
\newcommand{\N}{\ensuremath{\mathbb{N}}}
\newcommand{\T}{\ensuremath{\mathbb{T}}}
\newcommand{\eps}{\varepsilon}
\newcommand{\closure}[1]{\ensuremath{\overline{#1}}}

%MINE
%\newcommand{\norm}[1]{\left\Vert#1\right\Vert}
%\newcommand{\abs}[1]{\left\vert#1\right\vert}
\newcommand{\set}[1]{\left\{#1\right\}}
\newcommand{\epsi}{\varepsilon}
\newcommand{\To}{\longrightarrow}
\newcommand{\BX}{\mathbf{B}(X)}
\newcommand{\A}{\mathcal{A}}
\newcommand{\St}{\mathcal{S}}
\newcommand{\Par}{\mathcal{P}}
\newcommand{\La}{\mathcal{L}}
\newcommand{\F}{\mathcal{F}}

%LINEAR ALGEBRA
\newcommand{\tr}{\mathrm{tr}}
\newcommand{\sign}{\mathrm{sign}}
\providecommand{\norm}[1]{\lVert#1\rVert}

%PROBABILITY
\newcommand{\Var}{\mathrm{Var}\:}
\newcommand{\Cov}{\mathrm{Cov}\:}
\newcommand{\dis}{\mathrm{d}}

%Numerics
\newcommand{\Fb}{\textbf{F}}

%Theorem Styles
\newtheorem{thm}{Theorem}[section]
\newtheorem{cor}[thm]{Corollary}
\newtheorem{lem}[thm]{Lemma}

\theoremstyle{remark}
\newtheorem{rem}[thm]{Remark}

%Mattias
\newcommand{\mpar}[1]{{\marginpar{\tiny #1}}}
\newcommand{\notto}{\nrightarrow}
\newcommand{\Lloc}{L^1_{\mathrm{loc}}}
\newcommand{\Ito}{It{\^o}\ }
\newcommand{\Itos}{It{\^o}'s\ }
\newcommand{\eg}{e.g.\ }
\newcommand{\ie}{i.e.\ }
\newcommand{\as}{a.s.\ }
\newcommand{\iid}{i.i.d.\ }
\newcommand{\qand}{{\quad\text{and}\quad}}
\newcommand{\qor}{{\quad\text{or}\quad}}
\newcommand{\fixme}{{\begin{center}{\textbf{\huge fixme}}\end{center}}}
\newcommand{\work}{{\begin{center}{\textbf{\huge work}}\end{center}}}
\newcommand{\smilie}{{$:\hspace{-1mm}-\hspace{-0.5mm})$}}
\newcommand{\pa}{{\partial}}

%\newcommand{\acim}{\textsc{acim}\xspace}
%\newcommand{\acims}{\textsc{acim}s\xspace}

%%
%% Place here your \newtheorem's:
%%

%% Some examples commented out below. Create your own or use these...
%%%%%%%%%\swapnumbers % this makes the numbers appear before the statement name.
%\theoremstyle{plain}
%\newtheorem{thm}{Theorem}[chapter]
%\newtheorem{prop}[thm]{Proposition}
%\newtheorem{lemma}[thm]{Lemma}
%\newtheorem{cor}[thm]{Corollary}

%\theoremstyle{definition}
%\newtheorem{define}{Definition}[chapter]

%\theoremstyle{remark}
%\newtheorem*{rmk*}{Remark}
%\newtheorem*{rmks*}{Remarks}

%% This defines the "proo" environment, which is the same as proof, but
%% with "Proof:" instead of "Proof.". I prefer the former.
%\newenvironment{proo}{\begin{proof}[Proof:]}{\end{proof}}


\title{Procesos Estocásticos Otoño 2014, Trabajo \#1}
\author{Vidal Alcalá}

\begin{document}

\begin{center}
{\bf Procesos Estocásticos O. 2014 }

{\bf Trabajo individual \#1}
\end{center}

\bigskip

La notación 2-5 quiere decir el ejercicio 5 del capítulo 2 de \emph{Introduction to Probability Models}, eight edition, Sheldon M Ross .

\begin{enumerate}
\item Prueba que si $A,B,C\in \F$ entonces
\begin{equation*}
	\begin{split}
	P(A \cup B \cup C ) &= P(A) + P(B) + P(C)-P(A\cap B)-P(A\cap C)- P(B\cap C) + P(A\cap B\cap C) \:.
	\end{split}
\end{equation*}

\item Prueba que una v.a. en $(\Omega, \F ,P)$ es constante en cada conjunto de la partición $\Par$ inducida por la $\sigma$-álgebra $\F$.

\item Prueba que si $Y$ es una v.a. en $(\Omega,\F_X, P)$ y $\Omega$ es un conjunto discreto, entonces existe una función $g$ tal que $Y=g(X)$.

\item 1-17, 1-30, 1-32 , 1-40 , 1-45, 1-48

\item 2-11, 2-16, 2-27, 2-38 

\item Considera una variable aleatoria $T$ con función densidad
\begin{equation}\label{stopTime}
  \begin{split}
  f(t) &= \frac{1}{\sqrt{2\pi t^3}} e^{-1/2t},\qquad t>0\:.
  \end{split}
\end{equation}
\begin{enumerate}
\item Calcula la función de densidad de $X=\frac{1}{\sqrt{T}}$.
\item Genera 10,000 muestras independientes de $T$ usando la función \verb+ rnorm()+ y una transformación inversa. Compara el histograma de tus muestras con la función de densidad \eqref{stopTime} para comprobar que tus muestras tienen la distribución correcta. La solución debe estar contenida en un script con nombre \verb+ stop_time.R + .
\end{enumerate}


\end{enumerate}

\end{document}